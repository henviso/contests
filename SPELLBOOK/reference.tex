\documentclass[a4paper,oneside,twocolumn]{article}
\usepackage[utf8]{inputenc}
\usepackage[english]{babel}

\usepackage[T1]{fontenc}
\usepackage[sc]{mathpazo}
\linespread{1.185}

%\usepackage{eulervm}

\usepackage[landscape,left=.5cm,right=.5cm,top=2.1cm,bottom=.5cm]{geometry}

\setlength{\columnseprule}{0.25pt}
\usepackage{fancyhdr}
\pagestyle{fancy}
\fancyhf{}
\fancyhead[RE,RO]{Algorithm's Spellbook, page \bfseries\thepage}
\fancyhead[LE,LO]{henviso}

\usepackage{listings}

\usepackage{sectsty}
\sectionfont{\normalsize}

\lstset{ columns = fullflexible, basewidth = 0.45em, keepspaces= true }
\newcommand{\SECTION}[2]{\section*{#1} \addcontentsline{toc}{subsection}{#1} #2 \begin{center}\rule{400pt}{0.25pt}\end{center}}
\newcommand{\sourcefile}[1]{\begin{center}\textbf{#1}\end{center}\lstinputlisting{src/#1}}
\newcommand{\Csourcefile}[1]{\begin{center}\textbf{#1.cpp}\end{center}\lstinputlisting[language=C++]{src/#1.cpp}}

\usepackage{amsmath}
\usepackage{amsfonts}
\usepackage{amssymb}
\usepackage{amsthm}
\newcommand{\NN}{\mathbb{N}}
\newcommand{\ZZ}{\mathbb{Z}}
\newcommand{\QQ}{\mathbb{Q}}
\newcommand{\RR}{\mathbb{R}}
\newcommand{\then}{\Longrightarrow}
\newcommand{\floor}[1]{\left\lfloor#1\right\rfloor}
\newcommand{\ceil}[1]{\left\lceil#1\right\rceil}
\newcommand{\paren}[1]{\left(#1\right)}
\newcommand{\brackets}[1]{\left[#1\right]}
\newcommand{\braces}[1]{\left\{#1\right\}}
\newcommand{\abs}[1]{\left\lvert#1\right\rvert}
\newcommand{\nequiv}{\not\equiv}
\newcommand{\ds}{\displaystyle}
\newcommand{\bigO}{\mathcal{O}}
\newcommand{\norm}[1]{\abs{\abs{#1}}}
\newcommand{\degree}{\ensuremath{^\circ}}
\newcommand{\defun}[5] {
    \begin{array}{rrcl}
#1: & #2 & \longrightarrow & #3 \\
    & #4 & \longmapsto & #5
    \end{array}
}
\renewcommand{\le}{\leqslant}
\renewcommand{\ge}{\geqslant}
\renewcommand{\epsilon}{\varepsilon}

\newcommand{\stirfst}[2]{\genfrac{[}{]}{0pt}{}{#1}{#2}}
\newcommand{\stirsnd}[2]{\genfrac{\{}{\}}{0pt}{}{#1}{#2}}
\newcommand{\bell}[1]{\mathcal B_{#1}}
\newcommand{\seg}[1]{\overline{#1}}

\newcommand{\bmat}[1]{\begin{bmatrix}#1\end{bmatrix}}
\newcommand{\vmat}[1]{\begin{vmatrix}#1\end{vmatrix}}

\title{Algorithm's Spellbook}
\author{henviso}
\date{}

\begin{document}

\maketitle
\thispagestyle{fancy}

\begin{abstract}
This will someday become a very organized and structured notebook with nice algorithms and data structures to be used as reference...

\end{abstract}

\tableofcontents

\newpage
 
\newpage

\section{Macros}{
	\tiny{}
    \Csourcefile{macros}
}

\scriptsize{}

\SECTION{Basic}{
    \Csourcefile{basic}
    \SECTION{Recorrencia Linear}{
		\Csourcefile{reclog}
	}
	\SECTION{Strings Hash}{
		\Csourcefile{hashstrings}
	}
}

\section{Data Structures}{
	\SECTION{Arbitrary Precision Integers}{
		\Csourcefile{bigint}
	}
	\SECTION{Fenwick Tree}{
		\Csourcefile{fenwick}
	}
}

\section{Programação Dinâmica}{
	\SECTION{Lis Logarítmico}{
		\Csourcefile{lislog}
	}
	\SECTION{Knapsack}{
		\Csourcefile{knapsack}
	}
	\SECTION{Edit Distance}{
		\Csourcefile{editdist}
	}
}

\section{Graphs}{
	\SECTION{Maxflow EDMONDS-KARP}{
		\Csourcefile{fluxo}
	}
	
	\SECTION{Max Card Bip Matching}{
		\Csourcefile{MCBM}
	}
	
	\SECTION{Heavy Light Decomposition}{
		\Csourcefile{hld}
	}
	
	\SECTION{2 Sat}{
		\Csourcefile{2sat}
	}
	
}

\end{document}

